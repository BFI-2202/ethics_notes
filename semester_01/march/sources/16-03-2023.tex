\documentclass{article}
\usepackage[utf8]{inputenc}

\usepackage[T2A]{fontenc}
\usepackage[utf8]{inputenc}
\usepackage[russian]{babel}

\usepackage{multienum}
\usepackage{geometry}

\geometry{
    left=1cm,right=1cm,
    top=2cm,bottom=2cm
}

\title{Этика делового общения}
\author{Лисид Лаконский}
\date{March 2023}

\newtheorem{definition}{Определение}

\begin{document}
\raggedright

\maketitle
\tableofcontents
\pagebreak

\section{Этика делового общения — 16.03.2023}

Слово этикет французского происхождения, означает соблюдение норм
и правил.

В повседневной жизни человек, владеющий этикетом в любой ситуации
способен найти оптимальную линию поведения. Знание делового этикета, необходимое профессиональное качество любого бизнесмена. Особенно важен этикет при общении с зарубежными коллегами. А также знание этикета основа делового успеха.

В Европе деловой этикет зародился в эпоху абсолютизма, а в России понятие этикет вошло в начале 18 века (правление Петра 1). Но правила поведения, которые необходимо соблюдать зародились еще в Киевской Руси. В частности, они излагались в известном произведении Владимира Мономаха «Поучение детям», где он описывает нормы этикета, как нужно вести себя за столом, что старшие должны заботиться о младших, младшие уважать
старших и т.д.

Этикет – это система порядков, правил и норм социально-ролевого
общения.

\textbf{Функции этикета}:

\begin{multienumerate}
    \mitemxxx{Регламентирующая}{Символическая}{Коммуникативная}
\end{multienumerate}

\textbf{Виды этикета}:

\begin{multienumerate}
    \mitemxx{Светский (равенство субъектов, которые вступают в общение)}{Деловой (правила, принятые в сфере делового общения)}
    \mitemxxx{Служебный (служебная иерархия (подчиненный и начальник))}{Дипломатический}{Профессиональный и т.д.}
\end{multienumerate}

\textbf{Составляющие этикета}:

\begin{multienumerate}
    \mitemxxx{Внешний облик и одежда}{Манеры}{Правила поведения}
    \mitemx{Культура речи}
\end{multienumerate}

\subsection{Международный этикет}

Существует международный этикет, понятие возникло достаточно недавно.

Культурные нормы живут благодаря передаче традиций.

Обязательное и основное правило этикета – искреннее уважение к другому человеку. Это значит, что при первой встрече проявлять уважение к другому человеку, особенно если незнакомый, нельзя проявлять подозрительность и недоверие, другой человек будет это тонко чувствовать.

\subsubsection{Речевой этикет}

Существует вербальный этикет (словарный запас, манеры, стилистика речи). Важная норма речевого этикета – это готовность всегда ответить на заданный вопрос.

К \textbf{речевым этикетным нормам} относится:

\begin{enumerate}
    \item \textbf{Обращение} (должно быть всегда правильно выбранной формы, тональности, энергетики голоса, от этого во многом зависит дальнейшее взаимопонимание между людьми, к человеку следует обращаться по имени, при первой встрече обязательно запоминать имя собеседника, так вы быстрее его к себе расположите, но в России сохраняется обращение по имени и отчеству, у американцев практикуется обращение по имени, но нужно предупредить собеседника об этом, у немцев обращение либо по фамилии, либо по титулу. Если вы уже знакомы с деловым партнером (он старый друг), можно с ним разговаривать как с другом, но не переходить границу и не обижать его. В повседневной жизни, обращения могут быть самыми разнообразными, главное, чтобы они были не оскорбительными для человека. В настоящее время наиболее распространенными формами обращения к аудитории являются: дамы и господа, господа, уважаемые коллеги, дорогие друзья и т.д. Также можно обращаясь к официальному лицу, немного повышать его в должности (заместителя министра назвать господин министр) ему это будет приятно. Следует быть особенно внимательным в странах где при обращении сохраняются дворянские титулы (особенно это касается Англии)).
    \item \textbf{Приветствие} (первым здоровается мужчина с женщиной, дальше здоровается младший со старшим, подчиненный с начальником и т.д. На официальных приемах в первую очередь всегда приветствуют хозяйку и хозяина, затем дам, после этого старших по положению мужчин и затем остальных. Сидящий мужчина, когда приветствует даму или человека старше по званию, должен обязательно встать)
    \item \textbf{Жесты}, сопровождающие приветствие (наиболее распространенное рукопожатие (первой руку протягивает женщина мужчине, старший младшему, начальник подчиненному), поднятие руки, кивок головы, наклон, у женщин целуют ручку. Если мужчина приветствует на улице знакомую девушку, он должен немного приподнять свой головной убор, можно не приподнимать при -25. Если приветствие сопровождается рукопожатием, мужчина обязательно должен снять перчатку, женщина может не снимать, так как перчатка для женщины – часть женского туалета. Большое значение при приветствии имеет манера держаться (например, нельзя протягивая руку для приветствия, вторую держать в кармане, отводить глаза в сторону или продолжать разговор с другим человеком, это считается невоспитанностью и невежливостью). Не нужно шумно приветствовать собеседника, это некрасиво. Желательно, чтобы приветствие было развернутым и открытым, например: Добрый день Татьяна, как дела? )
    \item \textbf{Представление} (младший представляет старший, мужчину женщина. Существуют два способа знакомства: Знакомство через посредника – вас представляют. Второй способ Самостоятельное знакомство. Если это официальное знакомство, обязательно нужно указать профессию человека, положение и должность. Если знакомство без посредников нужно обращаться так: Разрешите (позвольте) с вами познакомиться (вам представиться). Если молодежная среда, то обычно достаточно сказать только имя. Если официальная встреча обязательно либо фамилию, либо фамилию и имя.)
    \item \textbf{Комплимент} (Приятные слова, несколько преувеличивающие положительные качества собеседника. Комплимент разделяют на два вида: светский (комплимент внешности и достоинств человека, обычно говорят знакомые, родственники и близкие люди. Считается, что сделать комплимент женщине проще чем мужчине. Комплимент всегда подчеркивает достоинство собеседника. Распространен в неофициальной обстановке. Всегда нужно благодарить за комплимент.) и деловой (обмен любезностями между сторонами партнерами. Взаимная обязательная процедура в деловой беседе. Существует письменный деловой этикет. Комплимент нужно говорить обязательно, но он должен быть правдивым, иначе вас заподозрят в неискренности.)
    \item \textbf{Поздравление} (признание значимости события или партнера. Поводы могут быть разными. Поздравление – знак внимания к человеку или организации.)
    \item \textbf{Сочувствие} (Может быть по поводу болезни, утраты, смерти близких людей. Некоторые люди считают, что сочувствовать – это лишний раз напоминать об утрате. Но на самом деле всегда говорят, что людям приятно чувствовать поддержку в трудной ситуации. Ученные заметили закономерность – если мужчина находится в стрессовом состоянии, то он пытается побыть в одиночестве, некоторое количество времени, чтобы успокоиться. А женщина наоборот нуждается выговориться кому-нибудь и быстро в себя приходит)
    \item \textbf{Прощание} (считается, что никогда нельзя прощаться навсегда (слово прощай лучше не говорить))
    \item \textbf{Подарки} (Выражение отношения к событию. Бывают подарки руководителю, их нужно делать от коллектива и только по торжественному поводу. Если хотите лично сделать подарок руководителю, лучше делать это не перед коллективом, к вам может испортиться отношение, подумают, что вы пытаетесь выслужиться. Подарки коллегам по работе – принцип ты мне, я тебе. Также лучше всего в столе иметь запас открыток и безделушек, которые можно подарить коллеге по работе. Подарки клиентам – подарок от фирмы, обычно это продукция, которую выпускает эта фирма. В качестве подарка официальным лицам или деловым партнерам, можно подарить хорошую книгу, альбом с репродукциями картин известных художников и т.д. Иногда можно дарить дорогие спиртные напитки, главное знать вкус человека. Женщинам дарят цветы, принято дарить нечетное кол-во цветов. Возможны индивидуальные подарки, в зависимости от пристрастий человека. Обязательно поблагодарить человека за подарок. Если подарок очень дорогой, и вы не можете подарить потом что-то по ценности похожее, лучше отказаться от такого подарка (Благодарю вас, я не могу себе это позволить, ненужно объяснять почему). Если вы отправляете подарки через курьера, обязательно должна быть визитная карточка.)
\end{enumerate}

\subsubsection{Официальные приемы}

На переговоры нужно приходить официально одетыми. Когда вы приезжаете на мероприятие, вы должны знать заранее где вам сесть, как это сделать. Уважаемый гость всегда садиться напротив главного принимающего лица. Справа садиться первый заместитель, слева второй заместитель. Если переговоры проходят с участием переводчика, то рассадка такая: уважаемый гость напротив главного принимающего лица, справа первый заместитель, слева переводчик. Гости должны всегда сидеть лицом к входной двери, либо к окну.

\textbf{Официальные приемы}:

\begin{enumerate}
    \item дневные (менее торжественны, чем вечерние) (завтраки, между 12 и 13 ч.)
    \item вечерние (“коктейль” - между 17-18) часами вечера, длятся около 2 часов; “ужин” - после 21 ч.)
\end{enumerate}

Обычно на приемах принято говорить тосты. Первый тост произносит хозяин приема. Дальше выступает главный гость. Остальные могут тосты не произносить.

\textbf{Правила этикета за столом}:

\begin{enumerate}
    \item Нельзя приступать к еде, пока этого не сделает хозяйка
    \item Мужчины должны подождать пока к еде не притронутся дамы
    \item После еды первой из-за стола встает хозяйка, потом все остальные
\end{enumerate}

Желательно не опаздывать на мероприятия.

\subsection{Деловая переписка}

Существует деловая переписка – разновидность официальной переписки.

Обязательно обратить внимание на правильность написания фамилии и
адреса человека, которому отправляете письмо. Деловые письма желательно писать на бланке организации. Подписывает такое письмо только руководство фирмы. Обязательно должна быть дата, день месяц год. Деловое письмо должно иметь безупречный внешний вид, быть на бумаге высшего качества, конверты для писем должны быть соответствующего размера и качества.

В письме должен прослеживаться доброжелательный тон. Негативную информацию лучше всего располагать в середине письма. Письмо должно быть написано либо на английском, либо на языке той страны куда оно отправляется. Письмо всегда складывается текстом внутрь. Если отправляете поздравления или соболезнования, письмо обязательно пишется от руки.

\end{document}