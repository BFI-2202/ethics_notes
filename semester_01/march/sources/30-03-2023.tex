\documentclass{article}
\usepackage[utf8]{inputenc}

\usepackage[T2A]{fontenc}
\usepackage[utf8]{inputenc}
\usepackage[russian]{babel}

\usepackage{multienum}
\usepackage{geometry}

\geometry{
    left=1cm,right=1cm,
    top=2cm,bottom=2cm
}

\title{Этика делового общения}
\author{Лисид Лаконский}
\date{March 2023}

\newtheorem{definition}{Определение}

\begin{document}
\raggedright

\maketitle
\tableofcontents
\pagebreak

\section{Этика делового общения — 30.03.2023}

\subsection{Европейский деловой этикет}

\subsubsection{Англия}

Англичане характеризуются деловитостью, почитанием собственности, традиций, законопослушания.

В беседах ценят умение слушать, а в деловых отношениях пунктуальность. Уважение к собеседнику. Всегда соблюдают формальность. Общение по имени только после специального разрешения. Используются рукопожатия. Но английские бизнесмены мало уделяют времени подготовке переговоров. Он наблюдателен, психологичен, способен скрыть профессиональную неподготовку.

Одежда: женщины - костюмы/платья, мужчины - костюм с галстуком.

О делах после работы разговаривать не принято.

Приглашение на обед: ксли пригласили, то отказываться нельзя, приходить в смокинге/вечернем платье. Напитки: джин/виски. Не принято произносить тосты и чокаться. Избегать разговоров о королевской семье, политике, финансах, личной жизни. Чай только с молоком, а не сливками. Также приглашают на ланч.

Если пригласили в дом - знак особого расположения.

Не принято обмениваться визитными карточками. Число 13 - часто отсутствует. Не принято дарить дорогие подарки.

\subsubsection{Франция}

Придают большое значение внешнему облику собеседника. На все официальные мероприятия мужчины - вечерние костюмы, женщины - вечерние платья; только из натуральных материалов и известных брендов. У женщин украшения только на мероприятия.

Ценят родную культуру, искусство. Любят спорить. Не прощают пренебрежительного отношения.

Телефонные переговоры не приемлят. Переговоры начинаются с 11:00. Не пунктуальны. Приветствуют рукопожатия. Пользуются визитными карточками. Не обращаться по имени (только если нет разрешения). Переговоры только на французском языке. Не стремиться перейти на английский.

Французская кухня - предмет гордости. Не оставлять еду на тарелке.
Подсаливать и добавлять специи (на столе) не принято.

Французы любят и предпочитают работать в одиночку. В разговоре не
принято касаться тем религии, семейного положения.

Если пригласили в дом, подарок: вино/шампанское.

Любят носить яркие рубашки. К деловому партнёру обращаются только
на Вы. Деловые подарки не приняты

\subsubsection{Германия}

Немцы бережливые, точные, свойственно гражданское мужество, сдержанность. Ценят профессионализм. Спешка неодобрительна.

Все встречи назначаются заранее. Проводятся с 10 до 16. Обращаться на Вы. Всегда присутствуют юристы. На всех руководящих постах пожилые люди.

Не принято приглашать делового партнёра домой, но если пригласили - большое уважение; подарок - вино, букет цветов (не розы) хозяйке.

Важны титулы, хозяйке титул мужа.

Пунктуальны. Не опаздывать. Не принято дарить деловые подарки.

Одежда строгая, мужчины - тёмный костюм/галстук; женщины - костюм (юбка, никаких брюк).

\subsubsection{Испания}

Человечные, с чувством юмора, способны работать в команде, эмоциональны.

В переговорах предпочитают на ты. Регламент встреч часто не соблюдается (любят говорить).

Приняты визитные карточки на испанском+английском.

Не принято приглашать домой, но если: подарок - цветы, конфеты и шампанское.

Не принято оставлять еду. Со спиртным - умеренность. Любят, когда хвалят их еду.

Хорошие собеседники, говорят на любую тему, но не принято: религия, режим Франко (2 мировая).

\subsection{Северо-американская деловая культура}

\subsubsection{США}

В дружбе не постоянны, дружат по интересам.

Не любят слишком официальную атмосферу. Одежда - может быть повседневная. Переговоры от 30 минут до 1 часа. Решение могут и менять.

Педантичны к оформлению документов.

Важна пунктуальность. Ценят трудолюбие, бережливость.

Считают, что знают весь деловой этикет других стран, что должны руководствоваться их (США) советами.

Способны бороться за доход, характерна напористость и агрессивность при заключении договоров.

\subsubsection{Канада}

Атмосфера бизнеса - европейская.

Пунктуальны. не любят, когда говорят о противоречиях между англо- и франкоязычным населением. Общительны о спорте.

Переговоры в ресторанах. Основные напитки - вино и пиво. При посещении дома - букет. Принято благодарить не за угощение, а за гостеприимство. Темы общения - любые, но не любят сравнение США и Канады. Обращаться на ты.

Одежда: высокого качества костюм/элегантная одежда.

\subsection{Деловая культура востока и арабских стран}

\subsubsection{Китай}

Уделяют много внимания сбору информации о предмете переговоров. Часто затягивают переговоры.

Только личные встречи (никаких телефонов).

Назначаются переговоры за 3 месяца. Пунктуальны. Нельзя откладывать переговоры.

Допустимы: рукопожатие, поклон.

Характерна сдержанность, уважение к старшим.

При обращении помним, что в Китае фамилия впереди имени (обращение по фамилии).

Принят обмен визитками. Разрешены тосты. Практикуется вручение
подарков. нельзя дарить часы.

\subsubsection{Япония}

Распространён коллективизм.

Носят чёрные (отличительный знак стажёров) или тёмно-синие костюмы. Прикреплён значок с эмблемой места работы. Чем выше ранг, тем строже одежда. Не следят за модой, главное в одежде - чистота (глаженность - не особо важна).

Принят обмен визитками (с небольшим поклоном).

Пунктуальны. Терпеливы. Стрессоустойчивы.

Приветствуются поклоны (меньше - рукопожатие). Чем ниже - тем больше уважения.

Рабочее время - больше всех в мире. Принято отказываться от отпуска.

Не принято приглашать домой, если вдруг - большое уважение. Не вытягивать руки/ноги. Не принято дарить цветы.

\end{document}