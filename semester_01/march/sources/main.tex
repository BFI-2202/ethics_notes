\documentclass[a4paper,14pt]{extarticle}
\usepackage[utf8]{inputenc}

\usepackage[T2A]{fontenc}
\usepackage[utf8]{inputenc}
\usepackage[russian]{babel}

\usepackage{multienum}
\usepackage{geometry}

\usepackage{cyrtimes}

\geometry{top=2cm}
\geometry{bottom=2cm}
\geometry{left=3cm}
\geometry{right=1cm}
\geometry{bindingoffset=0cm}

\title{Этика делового общения}
\author{Лисид Лаконский}
\date{March 2023}

\newtheorem{definition}{Определение}

\begin{document}
\raggedright

\maketitle
\tableofcontents
\pagebreak

\section{Этика делового общения — 02.03.2023}

Специфика делового общения заключается в том, что столкновением
взаимодействия экономических интересов и социальное регулирование осуществляется в правовых рамках. Чаще всего люди вступают в деловые отношения, чтобы юридически оформить взаимодействие в той или иной сфере.

В зависимости от различных признаков деловое общение делится на:

\begin{multienumerate}
    \mitemxxx{Устное/письменное общение}{Диалогическое/Монологическое}{Межличностная/Публичная}
    \mitemxx{Непосредственное/Опосредованное}{Контактное/Дистантное}
\end{multienumerate}

Деловое поведение – это осуществляемое в условиях рынка взаимодействие деловых людей с внешней средой (в основном с деловыми партнерами).

Характер и особенности данного взаимодействия определяются спецификой рыночной среды.

\subsection{Теории межличностного взаимодействия}

Существует несколько теорий объясняющих межличностное взаимодействие:

\begin{multienumerate}
    \mitemxx{Теория обмена}{Символический интеракционизм}
    \mitemxx{Теория управления впечатлениями}{Психоаналитическая теория}
\end{multienumerate}

\subsubsection{Теория обмена}

Согласно теории обмена каждый из нас стремится уравновесить возна-
граждение и затраты, чтобы сделать наше взаимодействие устойчивым и приятным. Эту теорию разработал известный американский социолог Джордж Хоманс. Согласно его теории люди стремятся взаимодействовать друг с другом только в том случае, когда в процессе взаимодействия происходит взаимовыгодный обмен важными ресурсами.

Хоманс выделяет основные принципы практической реализации теории
обмена:

\begin{enumerate}
    \item Поведение человека способствует или препятствует эффективному межличностному взаимодействию
    \item В ситуации, при которой вознаграждение за поведение зависит от определенных условий, человек будет стремиться воссоздать, чтобы эта ситуация была благоприятна для него.
    \item Уровень поощрения за реализацию взаимодействия
\end{enumerate}

\subsubsection{Теория символического интеракционизма Джорджа Мида}

Мид рассматривал любые действия человека, как социальное поведе-
ние, которое всегда основано на коммуникации. По его мнению люди все время взаимодействуют друг с другом, даже в отсутствии прямого контакта. По его заключениям люди реагируют не только на конкретные действия людей, но и на их мысленные намерения. Мид подчеркивал, что сущность символического интеракционизма заключается в том, что взаимодействие между людьми рассматриваются как непрерывный диалог, в процессе которого они наблюдают, осмысливают намерения друг друга и реагируют на них.

Таким образом, центральная идея данной концепции: Личность формируется всегда во взаимодействии с другими личностями.

\subsubsection{Теория управления впечатлениями}

Разработал Гоффман. Данный социолог жил в 20 веке.

По его мнению люди самостоятельно создают определенные ситуации,
которые оказывают благоприятное впечатление на других людей с целью осуществить что-то желаемое. По мнению Гоффмана социальные ситуации следует рассматривать как драматические спектакли в миниатюре. Т.е. люди ведут себя подобно актерам на сцене, используют определенные декорации чтобы создать впечатления о себе.

\subsubsection{Психоанализ Зигмунда Фрейда}

В его идеи лежит процесс взаимодействия опыта участников, связан-
ных детскими переживаниями. Фрейд говорил, что люди используют понятия, которые они освоили в раннем детстве. Для людей очень характерно почтительное отношение к человеку, который наделен финансовой, политической, экономической, организационной и другой властью. Также Фрейд выводит теорию покорности по отношению к лидеру.

\subsubsection{Этнометодология Гарфинкеля}

Суть теории: он рассматривает обыденные нормы, правила поведения смысла его в контексте повседневного социального взаимодействия людей.

Этнометодология обязательно использует методы социологического исследования, которая связана с описанием повседневных практик, социальных взаимодействий. Горфинкель первостепенное значение придавал социальной среде и взаимодействию людей. А также Горфинкель исследовал разные ситуации поведения людей. Например: поведение людей в суде, обычные беседы.

\subsubsection{Транзакционный анализ Эрика Берна}

Известная его книга «Люди и игры», широко используется в психотерапии, когда у людей наблюдаются психические расстройства. Психологи также активно используют ее, например, по коррекции поведения человека.

Эрик Берн вводит понятие «транзактный анализ – анализ взаимодействия». Ученый заметил, что мы в различных ситуациях занимаем различные позиции по отношению друг к другу. У каждого своя определенная позиция. Таким образом, суть теории Эрика Берна сводится к тому, что когда ролевые позиции партнеров согласованы, то их взаимодействие будет доставлять им чувство удовлетворения. Если есть несогласованность, то возникает конфликт.

\subsection{Манеры общения}

Каждый человек обладает своей своеобразной, неповторимой манерой общения. Типы:

\begin{enumerate}
    \item Доминантный собеседник – человек жесткий, напористый. Легко перебивает других людей.
    \item Недоминантный собеседник – человек уступчивый, легко теряется. Никогда не позволит себе перебить другого собеседника.
    \item Мобильный собеседник – всегда с легкостью переключается с других занятий на общение. Говорит быстро, иной раз торопливо. Проститься с ним также легко, как и завести беседу
    \item Ригидный - такому собеседнику некоторое время, чтобы включить в беседу. Он всегда внимательно слушает, основателен, говорит неспешно. Мысли излагает очень подробно. Попрощаться с ним сразу невозможно
    \item Экстраверт – этот человек всегда расположен к общению. Без общения скучает. Всегда любопытен ко всем сторонам жизни людей, даже своя жизнь не так интересует, как жизнь окружающих.
    \item Интроверт - не склонен к внешней коммуникации, для общения обычно выбирает до трех собеседников, которые обычно похожи на него самого.
\end{enumerate}

\subsection{Характеристика стратегий межличностного взаимодействия}

В повседневной жизни человек вступает во взаимодействие с другими людьми. Обязательно должен быть мотив. Ученые выделяют следующие мотивы:

\begin{multienumerate}
    \mitemxxx{Мотив общего выигрыша (мотив кооперации)}{Мотив собственного выигрыша}{Мотив относительного выигрыша}
    \mitemxx{Мотив выигрыша другого}{Мотив различий в выигрышах}
\end{multienumerate}

\end{document}