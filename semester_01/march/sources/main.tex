\documentclass[a4paper,14pt]{extarticle}
\usepackage[utf8]{inputenc}

\usepackage[T2A]{fontenc}
\usepackage[utf8]{inputenc}
\usepackage[russian]{babel}

\usepackage{multienum}
\usepackage{geometry}

\usepackage{cyrtimes}

\geometry{top=2cm}
\geometry{bottom=2cm}
\geometry{left=3cm}
\geometry{right=1cm}
\geometry{bindingoffset=0cm}

\title{Этика делового общения}
\author{Лисид Лаконский}
\date{March 2023}

\newtheorem{definition}{Определение}

\begin{document}
\raggedright

\maketitle
\tableofcontents
\pagebreak

\section{Этика делового общения — 02.03.2023}

Специфика делового общения заключается в том, что столкновением
взаимодействия экономических интересов и социальное регулирование осуществляется в правовых рамках. Чаще всего люди вступают в деловые отношения, чтобы юридически оформить взаимодействие в той или иной сфере.

В зависимости от различных признаков деловое общение делится на:

\begin{multienumerate}
    \mitemxxx{Устное/письменное общение}{Диалогическое/Монологическое}{Межличностная/Публичная}
    \mitemxx{Непосредственное/Опосредованное}{Контактное/Дистантное}
\end{multienumerate}

Деловое поведение – это осуществляемое в условиях рынка взаимодействие деловых людей с внешней средой (в основном с деловыми партнерами).

Характер и особенности данного взаимодействия определяются спецификой рыночной среды.

\subsection{Теории межличностного взаимодействия}

Существует несколько теорий объясняющих межличностное взаимодействие:

\begin{multienumerate}
    \mitemxx{Теория обмена}{Символический интеракционизм}
    \mitemxx{Теория управления впечатлениями}{Психоаналитическая теория}
\end{multienumerate}

\subsubsection{Теория обмена}

Согласно теории обмена каждый из нас стремится уравновесить возна-
граждение и затраты, чтобы сделать наше взаимодействие устойчивым и приятным. Эту теорию разработал известный американский социолог Джордж Хоманс. Согласно его теории люди стремятся взаимодействовать друг с другом только в том случае, когда в процессе взаимодействия происходит взаимовыгодный обмен важными ресурсами.

Хоманс выделяет основные принципы практической реализации теории
обмена:

\begin{enumerate}
    \item Поведение человека способствует или препятствует эффективному межличностному взаимодействию
    \item В ситуации, при которой вознаграждение за поведение зависит от определенных условий, человек будет стремиться воссоздать, чтобы эта ситуация была благоприятна для него.
    \item Уровень поощрения за реализацию взаимодействия
\end{enumerate}

\subsubsection{Теория символического интеракционизма Джорджа Мида}

Мид рассматривал любые действия человека, как социальное поведе-
ние, которое всегда основано на коммуникации. По его мнению люди все время взаимодействуют друг с другом, даже в отсутствии прямого контакта. По его заключениям люди реагируют не только на конкретные действия людей, но и на их мысленные намерения. Мид подчеркивал, что сущность символического интеракционизма заключается в том, что взаимодействие между людьми рассматриваются как непрерывный диалог, в процессе которого они наблюдают, осмысливают намерения друг друга и реагируют на них.

Таким образом, центральная идея данной концепции: Личность формируется всегда во взаимодействии с другими личностями.

\subsubsection{Теория управления впечатлениями}

Разработал Гоффман. Данный социолог жил в 20 веке.

По его мнению люди самостоятельно создают определенные ситуации,
которые оказывают благоприятное впечатление на других людей с целью осуществить что-то желаемое. По мнению Гоффмана социальные ситуации следует рассматривать как драматические спектакли в миниатюре. Т.е. люди ведут себя подобно актерам на сцене, используют определенные декорации чтобы создать впечатления о себе.

\subsubsection{Психоанализ Зигмунда Фрейда}

В его идеи лежит процесс взаимодействия опыта участников, связан-
ных детскими переживаниями. Фрейд говорил, что люди используют понятия, которые они освоили в раннем детстве. Для людей очень характерно почтительное отношение к человеку, который наделен финансовой, политической, экономической, организационной и другой властью. Также Фрейд выводит теорию покорности по отношению к лидеру.

\subsubsection{Этнометодология Гарфинкеля}

Суть теории: он рассматривает обыденные нормы, правила поведения смысла его в контексте повседневного социального взаимодействия людей.

Этнометодология обязательно использует методы социологического исследования, которая связана с описанием повседневных практик, социальных взаимодействий. Горфинкель первостепенное значение придавал социальной среде и взаимодействию людей. А также Горфинкель исследовал разные ситуации поведения людей. Например: поведение людей в суде, обычные беседы.

\subsubsection{Транзакционный анализ Эрика Берна}

Известная его книга «Люди и игры», широко используется в психотерапии, когда у людей наблюдаются психические расстройства. Психологи также активно используют ее, например, по коррекции поведения человека.

Эрик Берн вводит понятие «транзактный анализ – анализ взаимодействия». Ученый заметил, что мы в различных ситуациях занимаем различные позиции по отношению друг к другу. У каждого своя определенная позиция. Таким образом, суть теории Эрика Берна сводится к тому, что когда ролевые позиции партнеров согласованы, то их взаимодействие будет доставлять им чувство удовлетворения. Если есть несогласованность, то возникает конфликт.

\subsection{Манеры общения}

Каждый человек обладает своей своеобразной, неповторимой манерой общения. Типы:

\begin{enumerate}
    \item Доминантный собеседник – человек жесткий, напористый. Легко перебивает других людей.
    \item Недоминантный собеседник – человек уступчивый, легко теряется. Никогда не позволит себе перебить другого собеседника.
    \item Мобильный собеседник – всегда с легкостью переключается с других занятий на общение. Говорит быстро, иной раз торопливо. Проститься с ним также легко, как и завести беседу
    \item Ригидный - такому собеседнику некоторое время, чтобы включить в беседу. Он всегда внимательно слушает, основателен, говорит неспешно. Мысли излагает очень подробно. Попрощаться с ним сразу невозможно
    \item Экстраверт – этот человек всегда расположен к общению. Без общения скучает. Всегда любопытен ко всем сторонам жизни людей, даже своя жизнь не так интересует, как жизнь окружающих.
    \item Интроверт - не склонен к внешней коммуникации, для общения обычно выбирает до трех собеседников, которые обычно похожи на него самого.
\end{enumerate}

\subsection{Характеристика стратегий межличностного взаимодействия}

В повседневной жизни человек вступает во взаимодействие с другими людьми. Обязательно должен быть мотив. Ученые выделяют следующие мотивы:

\begin{multienumerate}
    \mitemxxx{Мотив общего выигрыша (мотив кооперации)}{Мотив собственного выигрыша}{Мотив относительного выигрыша}
    \mitemxx{Мотив выигрыша другого}{Мотив различий в выигрышах}
\end{multienumerate}

\section{Этика делового общения — 16.03.2023}

Слово этикет французского происхождения, означает соблюдение норм
и правил.

В повседневной жизни человек, владеющий этикетом в любой ситуации
способен найти оптимальную линию поведения. Знание делового этикета, необходимое профессиональное качество любого бизнесмена. Особенно важен этикет при общении с зарубежными коллегами. А также знание этикета основа делового успеха.

В Европе деловой этикет зародился в эпоху абсолютизма, а в России понятие этикет вошло в начале 18 века (правление Петра 1). Но правила поведения, которые необходимо соблюдать зародились еще в Киевской Руси. В частности, они излагались в известном произведении Владимира Мономаха «Поучение детям», где он описывает нормы этикета, как нужно вести себя за столом, что старшие должны заботиться о младших, младшие уважать
старших и т.д.

Этикет – это система порядков, правил и норм социально-ролевого
общения.

\textbf{Функции этикета}:

\begin{multienumerate}
    \mitemxxx{Регламентирующая}{Символическая}{Коммуникативная}
\end{multienumerate}

\textbf{Виды этикета}:

\begin{multienumerate}
    \mitemxx{Светский (равенство субъектов, которые вступают в общение)}{Деловой (правила, принятые в сфере делового общения)}
    \mitemxxx{Служебный (служебная иерархия (подчиненный и начальник))}{Дипломатический}{Профессиональный и т.д.}
\end{multienumerate}

\textbf{Составляющие этикета}:

\begin{multienumerate}
    \mitemxxx{Внешний облик и одежда}{Манеры}{Правила поведения}
    \mitemx{Культура речи}
\end{multienumerate}

\subsection{Международный этикет}

Существует международный этикет, понятие возникло достаточно недавно.

Культурные нормы живут благодаря передаче традиций.

Обязательное и основное правило этикета – искреннее уважение к другому человеку. Это значит, что при первой встрече проявлять уважение к другому человеку, особенно если незнакомый, нельзя проявлять подозрительность и недоверие, другой человек будет это тонко чувствовать.

\subsubsection{Речевой этикет}

Существует вербальный этикет (словарный запас, манеры, стилистика речи). Важная норма речевого этикета – это готовность всегда ответить на заданный вопрос.

К \textbf{речевым этикетным нормам} относится:

\begin{enumerate}
    \item \textbf{Обращение} (должно быть всегда правильно выбранной формы, тональности, энергетики голоса, от этого во многом зависит дальнейшее взаимопонимание между людьми, к человеку следует обращаться по имени, при первой встрече обязательно запоминать имя собеседника, так вы быстрее его к себе расположите, но в России сохраняется обращение по имени и отчеству, у американцев практикуется обращение по имени, но нужно предупредить собеседника об этом, у немцев обращение либо по фамилии, либо по титулу. Если вы уже знакомы с деловым партнером (он старый друг), можно с ним разговаривать как с другом, но не переходить границу и не обижать его. В повседневной жизни, обращения могут быть самыми разнообразными, главное, чтобы они были не оскорбительными для человека. В настоящее время наиболее распространенными формами обращения к аудитории являются: дамы и господа, господа, уважаемые коллеги, дорогие друзья и т.д. Также можно обращаясь к официальному лицу, немного повышать его в должности (заместителя министра назвать господин министр) ему это будет приятно. Следует быть особенно внимательным в странах где при обращении сохраняются дворянские титулы (особенно это касается Англии)).
    \item \textbf{Приветствие} (первым здоровается мужчина с женщиной, дальше здоровается младший со старшим, подчиненный с начальником и т.д. На официальных приемах в первую очередь всегда приветствуют хозяйку и хозяина, затем дам, после этого старших по положению мужчин и затем остальных. Сидящий мужчина, когда приветствует даму или человека старше по званию, должен обязательно встать)
    \item \textbf{Жесты}, сопровождающие приветствие (наиболее распространенное рукопожатие (первой руку протягивает женщина мужчине, старший младшему, начальник подчиненному), поднятие руки, кивок головы, наклон, у женщин целуют ручку. Если мужчина приветствует на улице знакомую девушку, он должен немного приподнять свой головной убор, можно не приподнимать при -25. Если приветствие сопровождается рукопожатием, мужчина обязательно должен снять перчатку, женщина может не снимать, так как перчатка для женщины – часть женского туалета. Большое значение при приветствии имеет манера держаться (например, нельзя протягивая руку для приветствия, вторую держать в кармане, отводить глаза в сторону или продолжать разговор с другим человеком, это считается невоспитанностью и невежливостью). Не нужно шумно приветствовать собеседника, это некрасиво. Желательно, чтобы приветствие было развернутым и открытым, например: Добрый день Татьяна, как дела? )
    \item \textbf{Представление} (младший представляет старший, мужчину женщина. Существуют два способа знакомства: Знакомство через посредника – вас представляют. Второй способ Самостоятельное знакомство. Если это официальное знакомство, обязательно нужно указать профессию человека, положение и должность. Если знакомство без посредников нужно обращаться так: Разрешите (позвольте) с вами познакомиться (вам представиться). Если молодежная среда, то обычно достаточно сказать только имя. Если официальная встреча обязательно либо фамилию, либо фамилию и имя.)
    \item \textbf{Комплимент} (Приятные слова, несколько преувеличивающие положительные качества собеседника. Комплимент разделяют на два вида: светский (комплимент внешности и достоинств человека, обычно говорят знакомые, родственники и близкие люди. Считается, что сделать комплимент женщине проще чем мужчине. Комплимент всегда подчеркивает достоинство собеседника. Распространен в неофициальной обстановке. Всегда нужно благодарить за комплимент.) и деловой (обмен любезностями между сторонами партнерами. Взаимная обязательная процедура в деловой беседе. Существует письменный деловой этикет. Комплимент нужно говорить обязательно, но он должен быть правдивым, иначе вас заподозрят в неискренности.)
    \item \textbf{Поздравление} (признание значимости события или партнера. Поводы могут быть разными. Поздравление – знак внимания к человеку или организации.)
    \item \textbf{Сочувствие} (Может быть по поводу болезни, утраты, смерти близких людей. Некоторые люди считают, что сочувствовать – это лишний раз напоминать об утрате. Но на самом деле всегда говорят, что людям приятно чувствовать поддержку в трудной ситуации. Ученные заметили закономерность – если мужчина находится в стрессовом состоянии, то он пытается побыть в одиночестве, некоторое количество времени, чтобы успокоиться. А женщина наоборот нуждается выговориться кому-нибудь и быстро в себя приходит)
    \item \textbf{Прощание} (считается, что никогда нельзя прощаться навсегда (слово прощай лучше не говорить))
    \item \textbf{Подарки} (Выражение отношения к событию. Бывают подарки руководителю, их нужно делать от коллектива и только по торжественному поводу. Если хотите лично сделать подарок руководителю, лучше делать это не перед коллективом, к вам может испортиться отношение, подумают, что вы пытаетесь выслужиться. Подарки коллегам по работе – принцип ты мне, я тебе. Также лучше всего в столе иметь запас открыток и безделушек, которые можно подарить коллеге по работе. Подарки клиентам – подарок от фирмы, обычно это продукция, которую выпускает эта фирма. В качестве подарка официальным лицам или деловым партнерам, можно подарить хорошую книгу, альбом с репродукциями картин известных художников и т.д. Иногда можно дарить дорогие спиртные напитки, главное знать вкус человека. Женщинам дарят цветы, принято дарить нечетное кол-во цветов. Возможны индивидуальные подарки, в зависимости от пристрастий человека. Обязательно поблагодарить человека за подарок. Если подарок очень дорогой, и вы не можете подарить потом что-то по ценности похожее, лучше отказаться от такого подарка (Благодарю вас, я не могу себе это позволить, ненужно объяснять почему). Если вы отправляете подарки через курьера, обязательно должна быть визитная карточка.)
\end{enumerate}

\subsubsection{Официальные приемы}

На переговоры нужно приходить официально одетыми. Когда вы приезжаете на мероприятие, вы должны знать заранее где вам сесть, как это сделать. Уважаемый гость всегда садиться напротив главного принимающего лица. Справа садиться первый заместитель, слева второй заместитель. Если переговоры проходят с участием переводчика, то рассадка такая: уважаемый гость напротив главного принимающего лица, справа первый заместитель, слева переводчик. Гости должны всегда сидеть лицом к входной двери, либо к окну.

\textbf{Официальные приемы}:

\begin{enumerate}
    \item дневные (менее торжественны, чем вечерние) (завтраки, между 12 и 13 ч.)
    \item вечерние (“коктейль” - между 17-18) часами вечера, длятся около 2 часов; “ужин” - после 21 ч.)
\end{enumerate}

Обычно на приемах принято говорить тосты. Первый тост произносит хозяин приема. Дальше выступает главный гость. Остальные могут тосты не произносить.

\textbf{Правила этикета за столом}:

\begin{enumerate}
    \item Нельзя приступать к еде, пока этого не сделает хозяйка
    \item Мужчины должны подождать пока к еде не притронутся дамы
    \item После еды первой из-за стола встает хозяйка, потом все остальные
\end{enumerate}

Желательно не опаздывать на мероприятия.

\subsection{Деловая переписка}

Существует деловая переписка – разновидность официальной переписки.

Обязательно обратить внимание на правильность написания фамилии и
адреса человека, которому отправляете письмо. Деловые письма желательно писать на бланке организации. Подписывает такое письмо только руководство фирмы. Обязательно должна быть дата, день месяц год. Деловое письмо должно иметь безупречный внешний вид, быть на бумаге высшего качества, конверты для писем должны быть соответствующего размера и качества.

В письме должен прослеживаться доброжелательный тон. Негативную информацию лучше всего располагать в середине письма. Письмо должно быть написано либо на английском, либо на языке той страны куда оно отправляется. Письмо всегда складывается текстом внутрь. Если отправляете поздравления или соболезнования, письмо обязательно пишется от руки.

\section{Этика делового общения — 30.03.2023}

\subsection{Европейский деловой этикет}

\subsubsection{Англия}

Англичане характеризуются деловитостью, почитанием собственности, традиций, законопослушания.

В беседах ценят умение слушать, а в деловых отношениях пунктуальность. Уважение к собеседнику. Всегда соблюдают формальность. Общение по имени только после специального разрешения. Используются рукопожатия. Но английские бизнесмены мало уделяют времени подготовке переговоров. Он наблюдателен, психологичен, способен скрыть профессиональную неподготовку.

Одежда: женщины - костюмы/платья, мужчины - костюм с галстуком.

О делах после работы разговаривать не принято.

Приглашение на обед: ксли пригласили, то отказываться нельзя, приходить в смокинге/вечернем платье. Напитки: джин/виски. Не принято произносить тосты и чокаться. Избегать разговоров о королевской семье, политике, финансах, личной жизни. Чай только с молоком, а не сливками. Также приглашают на ланч.

Если пригласили в дом - знак особого расположения.

Не принято обмениваться визитными карточками. Число 13 - часто отсутствует. Не принято дарить дорогие подарки.

\subsubsection{Франция}

Придают большое значение внешнему облику собеседника. На все официальные мероприятия мужчины - вечерние костюмы, женщины - вечерние платья; только из натуральных материалов и известных брендов. У женщин украшения только на мероприятия.

Ценят родную культуру, искусство. Любят спорить. Не прощают пренебрежительного отношения.

Телефонные переговоры не приемлят. Переговоры начинаются с 11:00. Не пунктуальны. Приветствуют рукопожатия. Пользуются визитными карточками. Не обращаться по имени (только если нет разрешения). Переговоры только на французском языке. Не стремиться перейти на английский.

Французская кухня - предмет гордости. Не оставлять еду на тарелке.
Подсаливать и добавлять специи (на столе) не принято.

Французы любят и предпочитают работать в одиночку. В разговоре не
принято касаться тем религии, семейного положения.

Если пригласили в дом, подарок: вино/шампанское.

Любят носить яркие рубашки. К деловому партнёру обращаются только
на Вы. Деловые подарки не приняты

\subsubsection{Германия}

Немцы бережливые, точные, свойственно гражданское мужество, сдержанность. Ценят профессионализм. Спешка неодобрительна.

Все встречи назначаются заранее. Проводятся с 10 до 16. Обращаться на Вы. Всегда присутствуют юристы. На всех руководящих постах пожилые люди.

Не принято приглашать делового партнёра домой, но если пригласили - большое уважение; подарок - вино, букет цветов (не розы) хозяйке.

Важны титулы, хозяйке титул мужа.

Пунктуальны. Не опаздывать. Не принято дарить деловые подарки.

Одежда строгая, мужчины - тёмный костюм/галстук; женщины - костюм (юбка, никаких брюк).

\subsubsection{Испания}

Человечные, с чувством юмора, способны работать в команде, эмоциональны.

В переговорах предпочитают на ты. Регламент встреч часто не соблюдается (любят говорить).

Приняты визитные карточки на испанском+английском.

Не принято приглашать домой, но если: подарок - цветы, конфеты и шампанское.

Не принято оставлять еду. Со спиртным - умеренность. Любят, когда хвалят их еду.

Хорошие собеседники, говорят на любую тему, но не принято: религия, режим Франко (2 мировая).

\subsection{Северо-американская деловая культура}

\subsubsection{США}

В дружбе не постоянны, дружат по интересам.

Не любят слишком официальную атмосферу. Одежда - может быть повседневная. Переговоры от 30 минут до 1 часа. Решение могут и менять.

Педантичны к оформлению документов.

Важна пунктуальность. Ценят трудолюбие, бережливость.

Считают, что знают весь деловой этикет других стран, что должны руководствоваться их (США) советами.

Способны бороться за доход, характерна напористость и агрессивность при заключении договоров.

\subsubsection{Канада}

Атмосфера бизнеса - европейская.

Пунктуальны. не любят, когда говорят о противоречиях между англо- и франкоязычным населением. Общительны о спорте.

Переговоры в ресторанах. Основные напитки - вино и пиво. При посещении дома - букет. Принято благодарить не за угощение, а за гостеприимство. Темы общения - любые, но не любят сравнение США и Канады. Обращаться на ты.

Одежда: высокого качества костюм/элегантная одежда.

\subsection{Деловая культура востока и арабских стран}

\subsubsection{Китай}

Уделяют много внимания сбору информации о предмете переговоров. Часто затягивают переговоры.

Только личные встречи (никаких телефонов).

Назначаются переговоры за 3 месяца. Пунктуальны. Нельзя откладывать переговоры.

Допустимы: рукопожатие, поклон.

Характерна сдержанность, уважение к старшим.

При обращении помним, что в Китае фамилия впереди имени (обращение по фамилии).

Принят обмен визитками. Разрешены тосты. Практикуется вручение
подарков. нельзя дарить часы.

\subsubsection{Япония}

Распространён коллективизм.

Носят чёрные (отличительный знак стажёров) или тёмно-синие костюмы. Прикреплён значок с эмблемой места работы. Чем выше ранг, тем строже одежда. Не следят за модой, главное в одежде - чистота (глаженность - не особо важна).

Принят обмен визитками (с небольшим поклоном).

Пунктуальны. Терпеливы. Стрессоустойчивы.

Приветствуются поклоны (меньше - рукопожатие). Чем ниже - тем больше уважения.

Рабочее время - больше всех в мире. Принято отказываться от отпуска.

Не принято приглашать домой, если вдруг - большое уважение. Не вытягивать руки/ноги. Не принято дарить цветы.

\end{document}