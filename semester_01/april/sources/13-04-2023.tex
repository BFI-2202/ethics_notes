\documentclass{article}
\usepackage[utf8]{inputenc}

\usepackage[T2A]{fontenc}
\usepackage[utf8]{inputenc}
\usepackage[russian]{babel}

\usepackage{multienum}
\usepackage{geometry}

\geometry{
    left=1cm,right=1cm,
    top=2cm,bottom=2cm
}

\title{Этика делового общения}
\author{Лисид Лаконский}
\date{April 2023}

\newtheorem{definition}{Определение}

\begin{document}
\raggedright

\maketitle
\tableofcontents
\pagebreak

\section{Этика делового общения — 13.04.2023}

Деловые мероприятия - общественные события в бизнесе:

\begin{multienumerate}
    \mitemxxx{Конференции}{Презентации}{Круглые столы}
    \mitemxxx{Форумы}{Семинары}{Встречи}
\end{multienumerate}

\subsection{Деловая встреча}

Деловая встреча - форма взаимодействия между людьми, которая предполагает наличие конкретной значимой цели. Это организованный тип общения между людьми.

\subsection{Пресс-мероприятия}

Пресс-мероприятия - встречи журналистов с представителями референтных групп (государственные учреждения, общественно-политические организации, коммерческие организации). Это эффективный метод передачи информации прессе и другим СМИ.

Причины созыва пресс-конференции:

\begin{enumerate}
    \item Есть серьёзный информационный повод
    \item Необходимо, чтобы о новом событии узнала целевая аудитория
    \item Если нужно, чтобы компания была упомянута в СМИ наряду с VIP-персоной
\end{enumerate}

\subsection{Деловые переговоры}

Деловые переговоры обязательно проводятся по определённым правилам и подчиняются собственным закономерностям. Главная цель - прийти к взаимовыгодному решению, избегнув конфликта.

Чтобы \textbf{правильно сформулировать цели деловых переговоров} надо знать:

\begin{multienumerate}
    \mitemxx{Интересы организации}{Положение организации на рынке}
\end{multienumerate}

Деловые переговоры бывают:

\begin{multienumerate}
    \mitemxx{Официальные (всё по протоколу)}{неофициальные (непринуждённая беседа, не подписываются бумаги)}
    \mitemxx{Внешние (с деловыми партнёрами и клиентами)}{внутренние (между сотрудниками))}
\end{multienumerate}

\textbf{Стадии переговоров}:

\begin{multienumerate}
    \mitemxxx{Подготовка}{Процесс переговоров}{Достижение согласия}
\end{multienumerate}

Считается, что если по итогу переговоров не был подписан контракт, то переговоров будто и не было.

\textbf{Техники ведения переговоров}:

\begin{enumerate}
    \item Тщательно избегать высказываний, оскорбляющих партнёра
    \item Не стоит игнорировать позицию собеседника
    \item Не следует делать замечания в ходе беседы
    \item В ходе переговоров возможны уточнения
    \item Избегайте перефразирования
    \item Не допускайте влияния своего эмоционального состояния на ход переговоров
    \item Необходимо правильно выбирать момент подведения промежуточных итогов
    \item Иногда переговоры ведутся нечестно. Так делать не следует
\end{enumerate}

\subsection{Деловое совещание}

Деловое совещание - общепринятая форма делового общения, когда обсуждаются вопросы, проблемы, которые требуют коллективного решения. В этом участвует коммуникативный лидер. На совещании рассматриваются только темы, которые не удаётся решить отдельным специалистам в рабочее время.

Обязательна повестка дня - письменный документ с темой совещания, перечнем обсуждаемых вопросов, время начала и окончания совещания, место проведения и должности докладчиков (обычно 6-7 человек), время рассмотрения каждого вопроса.

Лучше, чтобы совещание вёл не руководитель, а специалист, наиболее компетентный в проблемной области.

Рассаживать участников следует так, чтобы они видели глаза, лицо, мимику, жесты друг друга.

\subsection{Конференция}

Делится на дилерские, маркетинговые, отраслевые и т.д.

Следует разработать деловую креативную концепцию, арендовать помещение и оборудование, создать индивидуальный стиль, обеспечить проживание и питание участников.

Дилерское мероприятие - выездные конференции, выставки, съезды. Готовятся долго.

Тренинги - краткосрочные мероприятия обучающей направленности.

Выставки - с целью демонстрации продукта, имиджа компании, поддержки её репутации. Требуется подходящее помещение. После устраивают фуршеты, шоу. Могут быть постоянными или разовыми. Могут быть и для нескольких компаний.

\subsection{Презентация}

Презентация - для нахождения партнёров и клиентов, продвижения продукции. Требуется помещение с оборудованием.

\subsection{Деловые приёмы}

Деловые приёмы - светские мероприятия для налаживания контактов, связей с другими компаниями, решения определённых вопросов.

Бизнес-форумы. Масштабные мероприятия. Должна быть площадка, реклама, организован трансфер, проживание гостей.

\end{document}