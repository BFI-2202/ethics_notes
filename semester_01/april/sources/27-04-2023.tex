\documentclass{article}
\usepackage[utf8]{inputenc}

\usepackage[T2A]{fontenc}
\usepackage[utf8]{inputenc}
\usepackage[russian]{babel}

\usepackage{multienum}
\usepackage{geometry}

\geometry{
    left=1cm,right=1cm,
    top=2cm,bottom=2cm
}

\title{Этика делового общения}
\author{Лисид Лаконский}
\date{April 2023}

\newtheorem{definition}{Определение}

\begin{document}
\raggedright

\maketitle
\tableofcontents
\pagebreak

\section{Этика делового общения — 27.04.2023}

Деловое общение в значительной степени зависит от личных отношений участников коммуникаций, а также от морально-этических норм в коллективе.

Коллектив - группа людей, объединённых общими целями и мотивами. Всё чаще называют бизнес-команда - ограниченное количество психологически совместимых профессионалов, способных решать задачи и добиваться высоких результатов в профессиональной деятельности.

Важную роль в карьере менеджера играют краткосрочные перспективы
и амбициозность.

Руководству компании следует проводить регулярные аудиторские проверки, тем самым повышая для нечестных менеджеров риск быть разоблачённым.

Корпоративный климат формируется на основе этических норм и принципов.

Также в компании есть понятие рабочая группа - объединение из нескольких человек, которые взаимодействуют друг с другом для эффективного выполнения задачи.

\subsection{Специфика деловых отношений}

Работает по следующему принципу:

\begin{multienumerate}
    \mitemxx{Чёткое представление поставленной задачи}{Осознание своей роли в решении проблемы}
    \mitemxx{Объединение усилий каждого для достижения результата}{Поиск решений}
    \mitemx{Подготовка мероприятий, решающих проблему}
\end{multienumerate}

В любой рабочей группе есть психология.

Основа успеха любого коллектива - сотрудничество и взаимопомощь. Взаимодействия в группе зависят от психологического климата, способов общения, общего мнения и настроения.

Психологический климат - настроение коллектива, моральная и психологическая атмосфера, влияющая на взаимоотношения её участников. Здоровый климат повышает производительность работы, а неблагоприятный на 20\% снижает и вызывает нарушения безопасности.

\textbf{Типы взаимоотношений}:

\begin{enumerate}
    \item Приказание (Руководитель непрофессионален и не способен к ответственности)
    \item Внушение (Подчинённый несамостоятелен, но готов взять на себя ответственность)
    \item Участие (Подчинённый способен к самостоятельному выполнению задания и от руководителя требуется определённая поддержка и совместное принятие решений)
    \item Передача полномочий (У руководителя наименьшее участие, подсинённый достиг высокого уровня профессионализма)
\end{enumerate}

Типы темперамента (холерик, меланхолик, сангвиник, флегматик) следует учитывать.

\subsection{Лидерство}

В любом коллективе есть лидер.

Лидерство - способность оказывать влияние в группе для достижения цели. Лидер берёт ответственность за выполнение работ.

\begin{multienumerate}
    \mitemxx{Оценка лидера членами группы часто не совпадает}{Неточные цели приводят к конфликтам}
\end{multienumerate}

\textbf{Компоненты лидерства}:

\begin{enumerate}
    \item Деловое лидерство - способность руководить, сплачивать коллектив, решать определённые задачи.
    \item Эмоциональное лидерство - вызывает у людей уверенность.
    \item Информационное лидерство - человек-эрудит, может объяснить и помочь.
\end{enumerate}

Руководители \textbf{совмещают 6 основных ролей}:

\begin{multienumerate}
    \mitemxxx{Хозяин}{Предприниматель}{Дипломат}
    \mitemxxx{Менеджер}{Профессионал}{Член команды}
\end{multienumerate}

Моральный климат в коллективе в значительной мере формируется в деловом общении с подчинёнными. Наблюдая за руководителем они выясняют, какие действия обязательны, а какие нежелательны в коллективе.

Руководитель, который ставит определённые цели всегда будет стремиться сделать сплочённый коллектив. Человек почуствует себя психологически комфортно, когда пройдёт интеграцию в коллективе.

Руководитель может ругать подчинённых.

Не следует критиковать подчинённого при других подчинённых. Также нельзя давать советы подчинённым в личных делах.

Руководитель должен оказывать поддержку коллективу, доверять сотрудникам и защищать их.

Руководитель ответственен за перспективы своей организации.

Подчинённый может спорить с руководителем, но в рамках этики делового общения.

Два вида лидера: \textbf{формальный} и \textbf{неформальный}.

\end{document}

