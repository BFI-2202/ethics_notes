\documentclass{article}
\usepackage[utf8]{inputenc}

\usepackage[T2A]{fontenc}
\usepackage[utf8]{inputenc}
\usepackage[russian]{babel}

\usepackage{multienum}

\title{Этика делового общения}
\author{Лисид Лаконский}
\date{February 2023}

\newtheorem{definition}{Определение}

\begin{document}
\raggedright

\maketitle
\tableofcontents
\pagebreak

\section{Этика делового общения — 02.02.2023}

\subsection{Понятие деловой этики}

Термин «этика» относится к древнейшей науке, и относится к \textbf{5–4 векам до нашей эры}.

Родоначальником данного термина является \textbf{Аристотель}. Он образовал понятие «этический» (добродетель). Аристотеля занимала философская проблема: \textbf{«как следует поступать человеку, чтобы жить добродетельно?»}

\begin{definition}
\textbf{Этика} — область философии, предметом изучения которой являются проблемы морали

\textbf{Этика} — совокупность представлений о том, что есть добро, зло, справедливость, добродетель, нравственность

\textbf{Этика} — нравственные нормы, характеризующие ценности и нормы поведения определенных профессиональных групп
\end{definition}

Слово мораль относится к \textbf{философскому творчеству} и рассматривается непосредственно им. Проблемами этики и морали также занимался \textbf{Цицерон} Мораль происходит от слова \textbf{нравственный} и чаще всего понимается как \textbf{форма назидания в отношении норм поведения человека}. Этика рассматривается как \textbf{важнейшая система нормативной регуляции поведения человека в обществе}. Мораль рассматривается как синоним понятия \textbf{этика} и \textbf{нравственность}

\hfill

Важнейшим условием морального поведения служит присутствие в сознании индивида \textbf{нравственного идеала}

Этика \textbf{сформировалась} в эпоху, когда начал происходить распад родоплеменного строя и формирование раннего рабовладения.

К этому моменту общество накопило \textbf{обширные представления о добре и зле}. Эти представления \textbf{были выражены в древних эпосах}, например в \textbf{«Сказании о Гильгамеше»}

Также рассматривался вопрос смерти и бессмертия. Считалось, что \textbf{бессмертие человека — дела, которые он оставил после себя}

Учение вывели современное определение деловой этики (этики делового общения)

\begin{definition}
\textbf{Деловая этика} — это совокупность нравственных норм, правил и представлений, обеспечивающих регуляцию поведения людей в процессе производственной деятельности
\end{definition}

Деловая этика складывалась исторически в рамках традиционного общества, \textbf{определялась социальной структурой}, для которой была характерна роль \textbf{ритуалов, традиций, обычаев} и так далее.

Таким образом, \textbf{в традиционном обществе еще нет разрыва} между общественными моральными нормами и нормами деловой этики. \textbf{Этические ценности обладают самодостаточным значением}

Уже в Древней Греции заметное внимание уделялось изучению проблем деловой этики. Сократ также занимался вопросами этики, и считал, что \textbf{в основе делового общения лежит экономическая потребность}. Соответственно, для древних греков \textbf{статус начальника представлялся значительно выше, чем статус подчиненного}

\hfill

Для делового человека нового времени \textbf{характерно состояние внутренней психологической раздвоенности}: с одной стороны \textbf{он стремится к финансовой прибыли}, с другой стороны \textbf{он несет ответственность перед обществом (государством)}

Известный историк и экономист \textbf{Макс Вебер} также рассматривал вопросы этики, и считал, что \textbf{в основе раннего буржуазного общества лежит дух капитализма}

Для психологии современного делового человека \textbf{остается актуальна не только внутренняя раздвоенность}, но также и \textbf{проблема самоидентификации}

Известный философ Эрих Фромм в своей книге «Бегство от свободы» \textbf{определяет наличие этики в современном обществе}, а также \textbf{определяет человеческий тип как «личность с рыночным характером»}

В \textbf{современном крупном бизнесе} распространены \textbf{два кардинально разных подхода к вопросам деловой этики}:

\begin{enumerate}
    \item Сторонники \textbf{морального прагматизма} полагают, что \textbf{в бизнесе следование нормам морали лишь вредит интересам дела}. То есть, в данной среде тщательно избегают разговоров на темы нравственности и морали
    \item Сторонники противоположного подхода считают, что \textbf{нормы морали обязательны к исполнению в деловой сфере}. Очень часто данная позиция благотворно влияет на имидж и бренд компании, что в конечном счете приводит к ее финансовому процветанию
\end{enumerate}

\subsubsection{Основные категории этики}

\begin{multienumerate}
    \mitemxxx{Добро}{Зло}{Долг}
    \mitemxxx{Ответственность}{Честь}{Совесть}
    \mitemxxx{Достоинство}{Равенство}{Справедливость}
\end{multienumerate}

\textbf{Категориями} называются понятия, в которых \textbf{этика раскрывает сущность морали и моральной деятельности человека}

\textbf{Нормы} формируются на уровне общественного сознания, и это отражается категориями \textbf{добро} и \textbf{зло}

Такие категории как \textbf{совесть}, \textbf{честь}, \textbf{достоинство} характеризуют уровень \textbf{индивидуального морального сознания}

\subsection{Корпоративная культура как плод системы деловой культуры}

С конца 90-х годов XX века понятие \textbf{корпоративная культура} прочно вошло в лексикон отечественного бизнеса.

\subsubsection{Элементы корпоративной культуры}

\begin{multienumerate}
    \mitemxx{Приветствуемый стиль общения (в данной организации)}{Системность и регулярность менеджмента}
    \mitemxx{Система вознаграждений и поощрений}{Декларируемые ценности}
    \mitemxx{Регламентирующие документы}{Наличие и способы реализации корпоративных мероприятий}
\end{multienumerate}

\subsubsection{Уровни корпоративной культуры}

Понятие \textbf{уровней корпоративной культуры} впервые ввел известный ученый Эдгар Шейн в 1981 году.

Он выделяет \textbf{три уровня корпоративной культуры}:

\begin{enumerate}
    \item Поверхностный — внешние факты
    \item Внутренний — ценностные ориентации и верования
    \item Глубинный — базовые предположения
\end{enumerate}

\subsubsection{Факторы, вляющие на корпоративную культуру}

\textbf{Факторы, влияющие на корпоративную культуру}:

\begin{multienumerate}
    \mitemxx{Личности людей}{Личности топ-менеджмента и ключевых сотрудников}
    \mitemx{Внешнее окружение}
\end{multienumerate}

\subsubsection{Принципы формирования корпоративной культуры}

Основные \textbf{принципы формирования корпоративной культуры}:

\begin{multienumerate}
    \mitemxxx{Свобода}{Справедливость}{Общечеловеческие духовные ценности}
\end{multienumerate}

\textbf{Неэффективные} меры формирования корпоративной культуры:

\begin{enumerate}
    \item Административное насаждение правил и норм: введение системы штрафов, чрезмерный контроль за сотрудниками, устрашающие меры и так далее
    \item Назначение ответственных за создание корпоративной культуры
    \item Привлечение внешних специалистов
\end{enumerate}

Некоторые приемы реализации корпоративной культуры разрабатываются в различных организациях. Например, \textbf{размещение ценностей корпоративной культуры}, \textbf{особые традиции в компании}, \textbf{методы вдохновления сотрудников}, \textbf{обучение персонала профессиональным навыкам}

\subsubsection{Принципы ведения дел в России}

В корпоративной культуре выделяют \textbf{двенадцать принципов ведения дел в России}:

\begin{enumerate}
    \item \textbf{Принципы личности}:
    \begin{enumerate}
        \item Прибыль важнее всего, но честь дороже прибыли
        \item Уважай участников общего дела
        \item Воздерживайся от насилия или угрозы его применения
    \end{enumerate}
    \item \textbf{Принципы профессионала}:
    \begin{enumerate}
        \item Всегда веди дело сообразно средствам
        \item Опрадывай доверие — в нем ключ к успеху
        \item Конкурируй достойно
    \end{enumerate}
    \item \textbf{Принципы гражданина России}:
    \begin{enumerate}
        \item Соблюдай действующие законы и подчиняйся законной власти
        \item Для законного влияния объединяйся с единомышленниками
        \item Твори добро для людей и ничего не требуй за это взамен
    \end{enumerate}
    \item \textbf{Принципы гражданина земли}:
    \begin{enumerate}
        \item При создании и ведении дел не причиняй ущерба природе
        \item Найди в себе силы противостоять преступности и коррупции
        \item Проявляй терпимость к представителям другим культур, верований и стран
    \end{enumerate}
\end{enumerate}

\pagebreak
\section{Этика делового общения — 16.02.2023}

\subsection{Деловое общение}

\begin{definition}
    \textbf{Деловое общение} — процесс взаимосвязи и взаимодействия субъектов, при котором осуществляется обмен деятельностью, информации, опыта; целями которого являются:
    \begin{multienumerate}
        \mitemxx{Решение конкретной задачи}{Разрешение определенной проблемы}
        \mitemx{Достижения какой-то заявленной определенной цели}
    \end{multienumerate}
\end{definition}

Партнер в деловом общении выступает как личность, для всех его участников, которых отличает компетентность и взаимопонимание в обсуждаемых вопросах

Таким образом, главной целью делового общения является взаимовыгодное сотрудничество сторон

Основные формы делового общения:
\begin{multienumerate}
    \mitemxxx{Деловая беседа (возможно по телефону)}{Деловые переговоры}{Служебное совещание}
    \mitemxxx{Деловая дискуссия}{Пресс-конференция}{Публичная речь}
    \mitemx{Деловая переписка}
\end{multienumerate}

Ученые выдвигают технологию делового общения (то есть, эффективную общепринятую модель взаимодействия сторон при реализации форм современного делового общения)

Выделяют следующие технологии делового общения:
\begin{multienumerate}
    \mitemx{Вербальное деловое общение — при данной форме общения используется устная или письменная передача информации}
    \mitemx{Невербальное деловое общение — при передаче информации используется язык жестов, мимики, позы тела и так далее}
    \mitemx{Дистанционное деловое общение — общение посредством почты, телефона и так далее}
\end{multienumerate}

В результате делового общения происходит взаимодействие групп людей, и в науке выделяют следующие функции делового общения:

\begin{multienumerate}
    \mitemxxx{Информационно-коммуникативная — участники переговоров обмениваются определенной информацией}{Интерактивная — связана с процессом взаимодействия между участниками делового общения}{Перцептивная — выражается через процесс восприятия одного человека другим}
\end{multienumerate}

\subsubsection{Стили общения}

\begin{definition}
    Под \textbf{стилем общения} понимают комплекс личных качеств и действий лица, который в значительной мере предопределяет его отношение к определенной жизненной ситуации
\end{definition}

С этой точки зрения выделяют \textbf{ритуальный}, \textbf{манипулятивный} и \textbf{гуманистический} стили общения

\hfill

\textbf{Ритуальное общение} связано со стремлением индивида все сделать по определенным правилам и нормам

При \textbf{манипулятивном общении} партнер рассматривает своего собеседника в качестве инструмента для достижения определенной цели

\textbf{Гуманистическое общение} предполагает совместный поиск ответов на сложные вопросы. На первый план при этом стиле общения выходят духовные ценности

\hfill

Ключевую роль в деловом мире играет \textbf{невербальное} общение. Оно связано с особенностями человеческой психики и фиксирует наше внимание на внешних проявлениях собеседника

Именно невербальное общение как бы скрывает наиболее важную информацию в процессе деловых переговоров

Также существует \textbf{визуальное общение} — сфера контакта глаз. Визуальный контакт играет главную роль в невербальном общении

\subsubsection{Основные принципы делового общения}


\begin{multienumerate}
    \mitemxx{\textbf{Деловая репутация}. Роль деловой репутации огромна, поскольку она нарабатывается годами и активно влияет на деловой успех предпринимателя}{\textbf{Конкретность и четкость}}
    \mitemxx{\textbf{Взаимовыгодное сотрудничество}}{\textbf{Контроль над ситуацией}. Важно уметь держать себя в руках. Очень важно уметь контролировать свое поведение и бизнес}
    \mitemxx{\textbf{Умение слышать}. Необходимо инвестировать средства не только в сферу производства, но и в сферу обслуживания. Особенно важно \textbf{умение слушать и слышать своего клиента}}{\textbf{Умение сосредоточиться на главном}. Предпринимателю очень важно быстро и правильно расставлять приоритеты в своей деятельности}
    \mitemx{\textbf{Умение отделить личные отношения от бизнеса} считается одним из важных принципов деловой жизни}
    \mitemx{\textbf{Умение быть честным}. Этот принцип состоит в том, что необходимо соблюдать заключенные договоры и нести за них ответственность. Ради заключения какой-либо выгодной сделки нельзя манипулировать другими людьми, скрывать правду и так далее}
\end{multienumerate}

\subsection{Деловое общение как коммуникация}

\textbf{Деловое общение как коммуникации} — взаимодействие двух или более людей, целью которых является решение какой-либо проблемы

Деловая коммуникация представляет собой процесс \textbf{целесообразный} — то есть, вступая в контакт, коммуниканты преследуют определенные цели и интересы, которые могут совпадать между собой или, наоборот, вступать в противоречие

\hfill

В науке существует такое понятие как \textbf{коммуникативная компетентность} — она включает в себя умение адекватно вести себя в определенной ситуации и ставить определенные цели

Наиболее благоприятной формой коммуникации для убеждения собеседника является деловая беседа

\subsubsection{Типология барьеров в коммуникации}

\begin{enumerate}
    \item \textbf{Барьеры, обусловленные факторами среды}, то есть характеристики внешней физической среды, например, шум в помещении, за окном, ремонтные работы и так далее

    Негативное влияние усиливается, если в помещении плохая акустика

    Отвлекающей также может быть окружающая обстановка: яркое солнце, тусклый свет, цвет стен в помещении, пейзаж за окном и так далее

    А также температурные условия
    \item \textbf{Технические барьеры}: плохая телефонная связь, помехи в радиоэфире и так далее

    Технические барьеры в коммуникации связаны как с работой технических средств, так и могут быть обусловлены человеческим фактором 
    \item \textbf{Человеческие барьеры} — барьеры, причиной которых является сам человек. Например, эмоциональные барьеры, фонетический барьер (когда участники общения говорят на различных языках и диалектах)
    \item \textbf{Семантические барьеры} — проблема использования различных жаргонов, сленгов
    \item \textbf{Стилистический барьер} возникает при несоответствии стиля речи коммуникатора; тогда, когда информация передается функционально-книжным языком
\end{enumerate}

\subsubsection{Структурирование информации в деловом взаимодействии}

Существует \textbf{два основных приема структурирования информации в деловом взаимодействии}:

\begin{enumerate}
    \item Суть \textbf{правила рамки} состоит в том, что начало и конец любого делового разговора должны быть четко очерчены. То есть, в начале как правило сообщаются цели, намерения, перспективы, возможные результаты; а в конце обязательно должны быть подведены итоги, сделаны выводы

    Но в повседневном общении правило рамки достаточно часто нарушается
    \item \textbf{Правило цепи} направлено на внешнее структурирование общения. То есть, необходимые сведения должны так быть выстроены, что как бы соединены в цепь по каким-либо признакам
\end{enumerate}
    
Ученые доказали, что лучше всего запоминается фраза, состоящая из четырех—четырнадцати слов

\subsubsection{Вывод}

Для всех людей важно общаться таким образом, чтобы их правильно понимали; чтобы их слова не наталкивались на стену непонимания; чтобы их слушали и слышали

\end{document}