\documentclass{article}
\usepackage[utf8]{inputenc}

\usepackage[T2A]{fontenc}
\usepackage[utf8]{inputenc}
\usepackage[russian]{babel}

\usepackage{multienum}
\usepackage{geometry}

\geometry{
    left=1cm,right=1cm,
    top=2cm,bottom=2cm
}

\title{Этика делового общения}
\author{Лисид Лаконский}
\date{February 2023}

\newtheorem{definition}{Определение}

\begin{document}
\raggedright

\maketitle
\tableofcontents
\pagebreak

\section{Этика делового общения — 16.02.2023}

\subsection{Деловое общение}

\begin{definition}
    \textbf{Деловое общение} — процесс взаимосвязи и взаимодействия субъектов, при котором осуществляется обмен деятельностью, информации, опыта; целями которого являются:
    \begin{multienumerate}
        \mitemxx{Решение конкретной задачи}{Разрешение определенной проблемы}
        \mitemx{Достижения какой-то заявленной определенной цели}
    \end{multienumerate}
\end{definition}

Партнер в деловом общении выступает как личность, для всех его участников, которых отличает компетентность и взаимопонимание в обсуждаемых вопросах

Таким образом, главной целью делового общения является взаимовыгодное сотрудничество сторон

Основные формы делового общения:
\begin{multienumerate}
    \mitemxxx{Деловая беседа (возможно по телефону)}{Деловые переговоры}{Служебное совещание}
    \mitemxxx{Деловая дискуссия}{Пресс-конференция}{Публичная речь}
    \mitemx{Деловая переписка}
\end{multienumerate}

Ученые выдвигают технологию делового общения (то есть, эффективную общепринятую модель взаимодействия сторон при реализации форм современного делового общения)

Выделяют следующие технологии делового общения:
\begin{multienumerate}
    \mitemx{Вербальное деловое общение — при данной форме общения используется устная или письменная передача информации}
    \mitemx{Невербальное деловое общение — при передаче информации используется язык жестов, мимики, позы тела и так далее}
    \mitemx{Дистанционное деловое общение — общение посредством почты, телефона и так далее}
\end{multienumerate}

В результате делового общения происходит взаимодействие групп людей, и в науке выделяют следующие функции делового общения:

\begin{multienumerate}
    \mitemxxx{Информационно-коммуникативная — участники переговоров обмениваются определенной информацией}{Интерактивная — связана с процессом взаимодействия между участниками делового общения}{Перцептивная — выражается через процесс восприятия одного человека другим}
\end{multienumerate}

\subsubsection{Стили общения}

\begin{definition}
    Под \textbf{стилем общения} понимают комплекс личных качеств и действий лица, который в значительной мере предопределяет его отношение к определенной жизненной ситуации
\end{definition}

С этой точки зрения выделяют \textbf{ритуальный}, \textbf{манипулятивный} и \textbf{гуманистический} стили общения

\hfill

\textbf{Ритуальное общение} связано со стремлением индивида все сделать по определенным правилам и нормам

При \textbf{манипулятивном общении} партнер рассматривает своего собеседника в качестве инструмента для достижения определенной цели

\textbf{Гуманистическое общение} предполагает совместный поиск ответов на сложные вопросы. На первый план при этом стиле общения выходят духовные ценности

\hfill

Ключевую роль в деловом мире играет \textbf{невербальное} общение. Оно связано с особенностями человеческой психики и фиксирует наше внимание на внешних проявлениях собеседника

Именно невербальное общение как бы скрывает наиболее важную информацию в процессе деловых переговоров

Также существует \textbf{визуальное общение} — сфера контакта глаз. Визуальный контакт играет главную роль в невербальном общении

\subsubsection{Основные принципы делового общения}


\begin{multienumerate}
    \mitemxx{\textbf{Деловая репутация}. Роль деловой репутации огромна, поскольку она нарабатывается годами и активно влияет на деловой успех предпринимателя}{\textbf{Конкретность и четкость}}
    \mitemxx{\textbf{Взаимовыгодное сотрудничество}}{\textbf{Контроль над ситуацией}. Важно уметь держать себя в руках. Очень важно уметь контролировать свое поведение и бизнес}
    \mitemxx{\textbf{Умение слышать}. Необходимо инвестировать средства не только в сферу производства, но и в сферу обслуживания. Особенно важно \textbf{умение слушать и слышать своего клиента}}{\textbf{Умение сосредоточиться на главном}. Предпринимателю очень важно быстро и правильно расставлять приоритеты в своей деятельности}
    \mitemx{\textbf{Умение отделить личные отношения от бизнеса} считается одним из важных принципов деловой жизни}
    \mitemx{\textbf{Умение быть честным}. Этот принцип состоит в том, что необходимо соблюдать заключенные договоры и нести за них ответственность. Ради заключения какой-либо выгодной сделки нельзя манипулировать другими людьми, скрывать правду и так далее}
\end{multienumerate}

\subsection{Деловое общение как коммуникация}

\textbf{Деловое общение как коммуникации} — взаимодействие двух или более людей, целью которых является решение какой-либо проблемы

Деловая коммуникация представляет собой процесс \textbf{целесообразный} — то есть, вступая в контакт, коммуниканты преследуют определенные цели и интересы, которые могут совпадать между собой или, наоборот, вступать в противоречие

\hfill

В науке существует такое понятие как \textbf{коммуникативная компетентность} — она включает в себя умение адекватно вести себя в определенной ситуации и ставить определенные цели

Наиболее благоприятной формой коммуникации для убеждения собеседника является деловая беседа

\subsubsection{Типология барьеров в коммуникации}

\begin{enumerate}
    \item \textbf{Барьеры, обусловленные факторами среды}, то есть характеристики внешней физической среды, например, шум в помещении, за окном, ремонтные работы и так далее

    Негативное влияние усиливается, если в помещении плохая акустика

    Отвлекающей также может быть окружающая обстановка: яркое солнце, тусклый свет, цвет стен в помещении, пейзаж за окном и так далее

    А также температурные условия
    \item \textbf{Технические барьеры}: плохая телефонная связь, помехи в радиоэфире и так далее

    Технические барьеры в коммуникации связаны как с работой технических средств, так и могут быть обусловлены человеческим фактором 
    \item \textbf{Человеческие барьеры} — барьеры, причиной которых является сам человек. Например, эмоциональные барьеры, фонетический барьер (когда участники общения говорят на различных языках и диалектах)
    \item \textbf{Семантические барьеры} — проблема использования различных жаргонов, сленгов
    \item \textbf{Стилистический барьер} возникает при несоответствии стиля речи коммуникатора; тогда, когда информация передается функционально-книжным языком
\end{enumerate}

\subsubsection{Структурирование информации в деловом взаимодействии}

Существует \textbf{два основных приема структурирования информации в деловом взаимодействии}:

\begin{enumerate}
    \item Суть \textbf{правила рамки} состоит в том, что начало и конец любого делового разговора должны быть четко очерчены. То есть, в начале как правило сообщаются цели, намерения, перспективы, возможные результаты; а в конце обязательно должны быть подведены итоги, сделаны выводы

    Но в повседневном общении правило рамки достаточно часто нарушается
    \item \textbf{Правило цепи} направлено на внешнее структурирование общения. То есть, необходимые сведения должны так быть выстроены, что как бы соединены в цепь по каким-либо признакам
\end{enumerate}
    
Ученые доказали, что лучше всего запоминается фраза, состоящая из четырех—четырнадцати слов

\subsubsection{Вывод}

Для всех людей важно общаться таким образом, чтобы их правильно понимали; чтобы их слова не наталкивались на стену непонимания; чтобы их слушали и слышали

\end{document}